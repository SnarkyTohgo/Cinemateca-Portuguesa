A \mbox{\hyperlink{class_cinemateca}{Cinemateca}} Portuguesa pretende informatizar o sistema de compra de bilhetes para as sessões de cinema e eventos que promove, permitindo implementar descontos dedicados aos aderentes do Cartão Amigos da \mbox{\hyperlink{class_cinemateca}{Cinemateca}}. As instalações da \mbox{\hyperlink{class_cinemateca}{Cinemateca}} estão localizadas nas cidades do Porto e de Lisboa. Sobre cada localização da \mbox{\hyperlink{class_cinemateca}{Cinemateca}} o sistema deve guardar a morada, os eventos que promoveram nos últimos 12 meses e os aderentes associados à respectiva localização. Em relação aos aderentes, e de forma a garantir a emissão do Cartão Amigos da \mbox{\hyperlink{class_cinemateca}{Cinemateca}}, é importante guardar o nome, N\+IF, data de nascimento e ano de adesão.

O sistema deve incluir e permitir gerir informação relativa aos eventos da \mbox{\hyperlink{class_cinemateca}{Cinemateca}} planeados para os próximos 12 meses e deve permitir reservar bilhetes. Um evento é caracterizado por nome do evento, data e hora, duração, lotação máxima e preço do bilhete. A lotação máxima do evento pode ser pré definida (valor a atribuir quando da criação do evento) ou indefinida (valor=0, a ser determinado quando da alocação da sala).

Os aderentes do Cartão Amigos da \mbox{\hyperlink{class_cinemateca}{Cinemateca}} contam com um desconto equivalente aos anos de adesão ao Cartão \mbox{\hyperlink{class_cinemateca}{Cinemateca}} (num máximo de 15\%) sobre o valor do bilhete de qualquer evento promovido pela \mbox{\hyperlink{class_cinemateca}{Cinemateca}} Portuguesa.

A \mbox{\hyperlink{class_cinemateca}{Cinemateca}} dispõe de um conjunto de salas disponíveis para a realização de um evento. Uma sala é caracterizada por nome, capacidade da sala, horário de ocupação/disponibilidade. Antes de iniciar a venda de bilhetes para qualquer evento, deve ser alocada uma sala. Se o evento já possui lotação máxima, deve ser escolhida a sala disponível nesse horário, de menor capacidade que seja suficiente para esse evento (capacidade\+\_\+sala$>$= lotação\+\_\+máxima\+\_\+evento). Se o evento possui lotação indefinida, deve ser escolhida a sala disponível nesse horário de maior capacidade. Se não existir sala disponível, o evento não pode ser realizado.

Numa estratégia conjunta com a Câmara Municipal do Porto, foi decidido avançar com a opção de permitir que aderentes com residência na cidade do Porto e com mais de 65 anos assistam sem qualquer custo a eventos a realizar-\/se nas instalações da \mbox{\hyperlink{class_cinemateca}{Cinemateca}} na cidade do Porto. Esta opção apenas é acionada caso, faltando menos do que 8 horas para o início do evento, ainda não tenham sido vendidos mais do que 50\% dos bilhetes disponíveis para o mesmo. Nesse caso, os aderentes 65+ recebem uma mensagem com esta informação e podem aceitar ou não assistir ao evento.

Pretende-\/se monitorar informação sobre os eventos existentes (bilhetes comprados, valor das vendas dos bilhetes), clientes (compra de bilhetes, descontos) e salas.

Implemente também outras funcionalidades que considere relevantes, para além dos requisitos globais enunciados na página inicial. 